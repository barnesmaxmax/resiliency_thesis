\chapter{Conclusions and Recommendations}
\label{chp:chapter5}
\graphicspath{{figures/}{figures/chapter5/}}

\section{Overview}

This chapter summarizes the recommendations resulting from the resiliency
model application, it also contains information about obtaining the model and
outlines next steps.

\section{Problem and Objective}

UDOT is responsible for maintaining a
transportation system to promote public welfare and economic activity throughout
the state of Utah. UDOT is also responsible to maintain key components of the
national highway transportation system. Given the importance of this system,
UDOT has sought a way to identify those facilities which are most critical to
overall systemic function. Therefore, the objective of this study is to evaluate the relative systemic
criticality of highway links on a statewide network using a logit-based model
sensitive to changes in route path, destination choice, and mode choice.

The development of the presented model directly addresses the objective of
this study. The model is logit-based, incorporating a user's
ability to choose a new mode and destination, while still being sensitive to route
path in terms of time and distance. Using the logsum, which captures the total
value of the available choice set, different estimates were provided compared to
the travel time method, which is more traditional than the logsum method. The ability that the
presented model gives UDOT to quickly analyze any highway link on USTM, and return a
cost estimate, is highly advantageous. The model allows UDOT to determine
the relative systemic criticality of highway links on the network.

\section{Recommendations}

USTM is a trip-based model with a gravity trip distribution model and no mode
choice component. Using logit-based choice frameworks for trip distribution
and mode choice allows the model to incorporate greater sensitivity towards these
choicses, which in turn causes
the estimated costs associated with link closure to be different using the presented model.
Rather than forcing agents to travel long distances to their destinations after the
primary route becomes unavailable, the model
allows people to choose a new destination or mode if available.
Logit-based modeling returns smaller
estimations of the criticality of a link to a network than the travel time method
for most links.

It is recommended that USTM include a logit-based mode and destination
choice model in the future. This is standard practice in other states, and enables
more realistic modeling of
human behavior. Additionally, UDOT should consider behavior in its resiliency
analysis. Road users do change modes, destinations, and routes when a network
suddenly becomes altered, as has been observed in both the I-35W and I-85 disasters.
These decisions result in different criticality
priority rankings as shown through the logsum and travel time rankings in this study.
UDOT should also consider development of a more robust and detailed regional freight model.

\section{Limitations and Next Steps}

%feedback loop
Trip forecasting and estimation, such as that done in USTM and in the logsum model
can be prone to over or underestimation. Equilibrium feedback loops are used to
iterate through the trip distribution, mode and destination choice modules until the
impedance value, which is a measure largely based on congestion, stabilizes.
Trip assignment and a subsequent feedback loop was not included in the development
of the logsum model. Incorporating a
feedback loop would have allowed more accurate estimation network impedance and increased
travel times to be made. Accurate estimations would have allowed for more precise
route changes between OD pairs to be made considering congestion levels on the model
network. The decision to not include a feedback loop introduces limitations to the
ability of the model to accurately estimate the effects of link closure on
travel time and route choice due to congestion.

Inclusion of a feedback loop would cause the
travel time and logsum information to be fed back into the beginning of the
model and rerun continuously until the specifications of a convergence
parameter were met. The feedback loop
also allows for more accurate route choice, mode choice, and destination
choice to be made when the true effects of congestion are accounted for.
In June 2020, the UDOT Technical Advisory Committee decided to not include a
feedback loop to in order to save time
on model development for this study. A feedback loop and sensitivity analysis
should be included as part of future work.

%HBW flexibility
Another limitation in the logsum model is that all HBW trips are flexible in
destination choice. This implies that a user could choose to work in a different
TAZ than that in which their actual place of employment is located.
This might not be entirely logical for short-term highway closures, considering that
most workplace locations are fixed. With the recent turn towards telecommuting,
workplace location flexibility is likely more prevalent now than it has been in
the past. This increased flexibility would likely have a large impact on how HBW
trips are estimated in travel demand modeling. Thus, a more nuanced method for estimating
HBW trips that accounts
for both flexibility and inflexibility of workplace location should be developed.
Additional data could be collected in future travel
studies about workplace location could be gathered
and analyzed into a HBW model designed to model different types of trip---both flexible and inflexible.

It is not unreasonable to assume that in the event of an earthquake or
similar widespread disaster event, that multiple links could become
damaged. An important use of the presented model in future research would be
to analyze the results of simultaneous multiple link
loss. This will allow for
greater understanding of the effects that adverse events can have on Utah’s
highway network, and help UDOT to better prepare for the development and
maintenance needs of the future. The capability
is already incorporated into the structure of the model, however, a few modifications
need to be made in order for it to become fully functioning.
Additionally, the development of logsum calculations for trips with fixed
origins and destinations should also be considered. This model, however,
would be very different than the presented model because freight trips
do not typically have a mode or destination choice option, so it would be
more prudent to incorporate a method to account for other choices such as travel time,
distance or even elevation change on a given route. Driver awareness of physical
conditions of the highway network---including knowledge about current weather conditions,
road construction, or existing road closure---would also play a role in how a freight trip
is routed on a highway network. One option to overcome this limitation, is to use the national
freight network, which would allow freight trips to divert through other Interstate
facilites.

Another limitation present in the travel time method of analysis is that passenger
trips occurring between internal-external or external-internal nodes were not
included. These trips likely do not make up a large portion of trips on the network,
however, this may still affect the ability of the travel time method of comparison
to make estimations that reflect all trips on the network.

\section{Summary}

The development of a logit-based travel demand model can improve the way UDOT
estimates costs associated with link loss when compared
with traditional modeling methods. The presented logsum model provides a different cost
estimate than the travel time method. This is due to the ability of the logsum
model to account for user mode and destination choice. Additionally,
logit-based models possess unique properties which allow modelers
to incorporate multiple types of data. With this ability in mind, the presented logsum model
includes several types of available data deemed pertinent to
constructing a working logit-based model in Utah.

The logsum cost estimations account for user behavior which can help professionals
better evaluate systemic criticality of Utah's infrastructure. Logsums are becoming an increasingly
important consideration in modern modeling practice because of their unique
ability to account for user choice while estimating
network trips. The presented logsum model generates valuable results that should
be used to prioritize link criticality to the overall functionality of Utah's
highway network. There are several important implications that follow the
results. First, consideration of only the change in travel time as the main measure
of dis-benefit may not accurately represent some situations. HBW, NHB, and HBO trips,
can be better estimated using the logsum because these trip purposes have more flexible
mode and destination choices than REC or freight trips do. However, both REC and
freight trips were only considered by the travel time method because they typically
have fixed OD pairs. Availability of data which could be used to better determine how
REC or freight trips might change mode or destination choice could be used to better
estimate costs for these purposes. Second, allowing users in the model
to choose both a new mode and destination increases the resemblance of a real life
decision making process. Incorporating a method to better represent human behavior
can only help a model to more accurately reflect what researchers have observed
after the I-35W and I-85 incidents. Importantly, the presented model is able to provide results produced
using two different methods of cost estimation, which can help UDOT to understand
how different modeling methods provide more or less conservative estimates of criticality.
However, combining estimation methods into a single model may be effective because trip purposes which
are flexible or inflexible can be estimated at the same time.

It is evident that the presented model brings several
modeling estimation advantages to the forefront of current research, something which must be
explored in the future. The biggest contribution of this thesis to
the field of transportation modeling is the incorporation of a logit-based model
into a statewide highway network. Applying a logsum to a network as large as USTM, and
estimating the dis-benefit associated with link closure, provide the basis for both
UDOT and other state transportation agencies to be able to quickly estimate
the criticality of any single link or combination of links to overall systemic
resiliency of a highway network while accounting for route change, mode choice, and destination choice.
