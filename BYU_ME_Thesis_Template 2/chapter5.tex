\chapter{Conclusions and Recommendations}
\label{chp:chapter5}
\graphicspath{{figures/}{figures/chapter5/}}

%TODO update all references
%TODO import figures
%TODO figure out how to use TARGETS on tables
\section{Overview}

This chapter summarizes the recommendations resulting from the resiliency
model application and contains information about obtaining the model and
outlines next steps.

\section{Recommendations}

The USTM model is a gravity-based travel demand model, while the resiliency
model is logit-based. The logit-based nature of the resiliency model allows
for greater sensitivity in user mode and destination choice, which causes
the estimated costs associated with link closure to be lower yet more
accurate for estimations made using the resiliency model. The resiliency
models incorporation of both the logsum for HBW, HBO, and NHB purposes, as
well as the travel time calculation for the other purposes included in USTM
provide important functionality towards estimating the true costs
associated with long term highway closure.

The recommendations resulting from the adaptation of a logit-based model on
the USTM network are that logit-based modeling returns more sensitive
estimations of the value of a link in the network. This is a highly
important outcome because more accurate and efficient estimation allows
UDOT to better understand the monetary importance of highway links
throughout Utah. Additionally, the model’s design allows for further
analysis of additional link closure, eased identification of critical
points on Utah's highway network, and even multi-link closure in the
future.

\section{Obtaining the Model and it's Documentation}

The model is hosted in an online GitHub repository which is maintained by
the BYU Transportation Lab. The repository can be found here:

https\://github.com/byu\-transpolab/ustm\_resiliency

\section{Limitations and Next Steps}

Overall implementation of a logit-based travel demand model that more
accurately captures demand on the USTM network is a highly important tool
for continued development. Updating USTM to include a logit-based travel
demand model instead of a gravity-based model would increase the accuracy
of travel demand estimates moving forward. This, in turn, would allow for
better travel estimates between TAZ across the state, and would likely
cause more accurate estimations of traffic volumes on the road network to
be estimated, and therefore future development and maintenance activities
could be adjusted to accomodate future travel needs in Utah.  Updating USTM
to have a logit-based model would increase overall highway network capacity
to accurately estimate travel demand on Utah’s road network.

The non-motorized (NMOT) distance threshold was originally set at 50 miles
because we believed that this would cap the most extreme trips but still
allow the possibility of longer NMOT trips. NMOT trips are heavily
penalized per each additional mile, but NMOT trips have improved access to
short- and medium-range destinations, which clearly creates a problem when
trips become longer and can be assigned NMOT as a mode. Setting the
threshold at 50 miles was a mistake; the 90th percentile walk and cycling
trips in the Utah Household Travel Survey (UHTS) are less than 2 and 5
miles respectively. In order to better represent NMOT trips, an upper
threshold of 2.5 miles was assigned. Additionally, the NMOT skim was held
constant throughout each iteration. This was done because pedestrians and
cyclists typically have better access to side streets, or are more likely
to find shorter routes not represented on the USTM network when presented
with a closed link.

Adding a congestion feedback loop to the resiliency model would allow more
accurate cost estimations to be made. A feedback loop would cause the
travel time and logsum information to be fed back into the beginning of the
model and rerun continuously until the specifications of a convergence
parameter were met. The loop allows for better estimates of the true change
in travel time due to both route change and increased congestion. The loop
also allows for more accurate route choice, mode choice, and destination
choice to be made when the true effects of congestions are accounted for.
In June 2020, the TAC decided to not include a feedback loop to save time
on model development for Phase 1. A feedback loop and sensitivity analysis
will be included as part of Phase 2.

It is not unreasonable to assume that in the event of an earthquake or
similar widespread disaster event, that multiple links could become
damaged. The ability to analyze the results of simultaneous multiple link
loss will be included as a part of Phase 2 as well. This will allow for
greater understanding of the effects that adverse events can have on Utah’s
highway network, and help UDOT to better prepare for the development and
maintenance needs of the future.

\section{Summary}

The development of a logit-based travel demand model can improve the
ability of UDOT to accurately estimate the costs per day of link loss that
Utahn’s would experience. The resiliency model provides sensitive
estimates that more accurately represent the costs associated with link
closure than a travel time increase methodology by itself would be able to
capture. Using the logsum, estimations sensitive to user choice were
found, which can help professionals to better evaluate risk to Utah’s
infrastructure. User choice is highly important in modern modeling
practices because user choice allows a model to estimate information more
precisely and accurately. The information provided by the resiliency model
should be used to prioritize link importance to the functionality of
Utah’s highway network. The resiliency model more accurately estimates
costs experienced by Utahn’s due to link loss than traditional methods for
determining costs.
