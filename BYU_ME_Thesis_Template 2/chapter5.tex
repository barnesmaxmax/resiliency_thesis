\chapter{Conclusions and Recommendations}
\label{chp:chapter5}
\graphicspath{{figures/}{figures/chapter5/}}

\section{Overview}

This chapter summarizes the recommendations resulting from the resiliency
model application, it also contains information about obtaining the model and
outlines next steps.

\section{Recommendations}

The USTM model is a gravity-based travel demand model, while the Resiliency
Model is logit-based. The logit-based nature of the Resiliency Model allows
for greater sensitivity in user mode and destination choice, which causes
the estimated costs associated with link closure to be lower using the Resiliency Model. The Resiliency
Model's incorporation of both the logsum for HBW, HBO, and NHB purposes, as
well as the travel time calculation for the other purposes included in USTM
provide important functionality towards estimating more realistic, conservative costs
associated with long term highway closure.

Logit-based modeling returns more conservative
estimations of the value of a link in the network. This is a highly
important adaptation to USTM because more accurate and efficient estimation allows
UDOT to better understand the monetary importance of highway links
throughout Utah. Additionally, the model’s design allows for further
analysis of additional link closure, eased identification of critical
points on Utah's highway network, and even multi-link closure in the
future. Thus, it is recommended that USTM include a logit-based mode and destination
choice model in the future.

\section{Limitations and Next Steps}

Implementation of a logit-based travel demand model is highly important for
resource allocation moving into the future. Updating USTM to include a
logit-based trip distribution model instead of a gravity-based model would
improve the flexibility of travel demand estimates moving forward.

The decision to not include a feedback loop, which iteratively runs the model
until a the specifications of a convergence factor are met. Incorporating a
feedback loop would have allowed for more accurate estimation of increased
travel times and more accurate route choices between OD pairs in the final
iteration of the Resiliency model compared with the first iteration.
This decision introduces limitations on the
ability of the model to accurately estimate the effects of link closure on
travel time. This limits the ability of the model to estimate accurate changes in
response to increased travel time due to congestion because cannot be
accurately measured after one iteration.

Adding a congestion feedback loop to the Resiliency Model would allow more
accurate cost estimations to be made. A feedback loop would cause the
travel time and logsum information to be fed back into the beginning of the
model and rerun continuously until the specifications of a convergence
parameter were met. The loop would allow for better estimates of the change
in travel time due to both route change and resulting congestion. The loop
also allows for more accurate route choice, mode choice, and destination
choice to be made when the true effects of congestions are accounted for.
In June 2020, the UDOT Technical Advisory Committee decided to not include a
feedback loop to save time
on model development for this study. A feedback loop and sensitivity analysis
should be included as part of future work.

It is not unreasonable to assume that in the event of an earthquake or
similar widespread disaster event, that multiple links could become
damaged. An important use of the Resiliency Model in future research would be
to analyze the results of simultaneous multiple link
loss. This will allow for
greater understanding of the effects that adverse events can have on Utah’s
highway network, and help UDOT to better prepare for the development and
maintenance needs of the future. Developing a fully functioning transit oriented logit model
would be highly advantageous to the Utah Transit Authority as well. The capability
is already incorporated into the structure of the resiliency model, however, a few modifications
need to be made in order for it to become fully functioning.
Additionally, the development of logsum calculations for trips with fixed
origins and destinations should also be considered. This model, however,
would be very different than the Resiliency Model because freight trips
do not typically have a mode or destination choice option, so it would be
more prudent to be able to account for other choices such as travel time,
distance or even elevation change on a given route. One option is to use the national
freight network, which would allow freight trips to divert through other Interstate
facilites.



\section{Summary}

The development of a logit-based travel demand model can improve the way UDOT
estimates costs associated with link loss experienced by Utahns than do traditional modeling methods. The Resiliency Model provides different
estimates that represent the costs associated with link
closure than a travel time increase methodology by itself would be able to
capture. The unique properties that logit-based models offer allow multiple types
of data to be included. As such, the Resiliency Model includes available data which
was pertinent for constructing a working model in Utah. The Resiliency Model logsum estimations, which account for user choice, were
found and can help professionals better evaluate risk to Utah’s
infrastructure. User choice is a highly important consideration in modern modeling
practices because it allows a model to estimate information while accounting for the ability of a user to choose. The information
provided by the Resiliency Model
should be used to prioritize link importance to the functionality of
Utah’s highway network. There are several implications of this. First, purely
using change in travel time or travel time delay as the main dis-benefit may not
be entirely accurate in most situations. Second, allowing users in the model
to choose both a mode and destination, increases the resemblance of a real life
decision making process, especially given adverse circumstance on Utah's
highway network. Importantly, the Resiliency Model is able to provide results measured
using two different methods of cost estimation, making results comparable, while allowing different priorities to be evaluated. It is evident that the Resiliency Model provides several
modeling estimation advantages to the forefront, something which must be
explored further in future research.
