\chapter{Introduction}
\label{chp:chapter1}
\graphicspath{{figures/}{figures/chapter1/}}

\section{Problem Statement}
UDOT is responsible for maintaining a
transportation system to promote public welfare and economic activity throughout
the state of Utah. UDOT is also responsible to maintain key components of the
national highway transportation system. Given the importance of this system,
UDOT has sought a way to identify those facilities which are critical to smooth
operation of the system.

In 2017, a group called \citet{aem2017} completed a risk and resilience analysis report for the I-15 corridor on behalf of
UDOT. This analysis quantified risk as the probability of threats (earthquakes, floods, fires,
etc.) multiplied by the criticality of the asset to the overall system. The AEM
analysis has two primary limitations. First, the methods are proprietary to
AEM such that UDOT cannot now apply the methods to study the criticality of other transportation
corridors with regional and national significance (e.g., U.S. Route 6, I-70, I-80). But more
importantly, the current index treats each UDOT asset – each bridge, highway segment, etc. – as an
independent unit, when in fact UDOT operates a system of interrelated transportation facilities. The criticality
of a single bridge to the overall system is not determined by the volume of traffic it supports
directly, but by how inconvenient it would be for that traffic to find another path or destination,
were the bridge to fail. Resiliency must therefore be considered a function of network, mode, and destination
alternatives which comprise systemic redundancy. Developing a model capable of accounting for the choices a user makes will help
transportation planners to calculate sensitive estimates of the costs associated with link closure.

\section{Objectives}
The primary objective of this study is to develop a methodology and tool to evaluate the
relative systemic criticality of highway links on a statewide network using a logit-based model
sensitive to changes in route path, destination choice, and mode choice.
This tool is based on data collected
from USTM, with certain improvements and additional model
features to more accurately capture the economic costs associated with an impaired state highway
network. In particular, we develop a method that explicitly considers the availability of
alternative destinations, modes, and routes to individuals traveling on the impaired network. A
secondary objective of this research is to apply the model to evaluate the criticality of
specific
highway links in Utah, by comparing the change in accessibility, or dis-benefit,
experienced by road users.
This thesis presents the results of this evaluation applied
on 40 individual highway links.

\section{Scope}
The purpose of this thesis is to provide a functional tool that can be used to evaluate
the potential economic costs associated with highway link closure in Utah. USTM comprises
the entire highway network in Utah, with about 75,000 links, 36,000 nodes and 8,500 Transportation Analysis Zones (TAZ).
Additionally, USTM covers the geographic area in which about 3.2 million people live.
Developing a choice model (aptly named the Resiliency Model) can help to determine
the effects of road closures or long term link loss for the entire State of Utah.
To better determine these effects, the Resiliency Model is based on the theory of logit choice modeling and
shortest path finding in a network. The specific choice utility equations in the model represent
a plausible utility outcome, but the focus of this research has not been on developing robust
utility equations or calibrated volume-delay functions. This model is therefore not designed to
forecast traffic volumes nor is it designed for any purpose other than providing a comparative estimate
of the effects of link loss by man-made or natural causes. Creating a choice model
which uses the same network as USTM, and incorporates multiple datatypes is
advantageous to UDOT moving forward because of the alternative estimates a choice
model can provide.

\newpage
\section{Outline of Report}

\noindent This thesis is organized as follows:

\begin{description}
	\item Chapter 1.	This introductory chapter.
	\item Chapter 2.	This chapter presents a Literature Review, summarizing previous attempts to model network resiliency using the choices and accessibility of individuals on the impacted network.
	\item Chapter 3.	This chapter presents a proposed model design and implementation of the model within the CUBE transportation planning software application. This chapter also describes model calibration efforts.
  \item Chapter 4.	This chapter presents the Model Application, description and comparison of model results, from the model developed in Chapter 3, to the highway links identified in Chapter 4.
	\item Chapter 5.	This chapter presents the conclusions, summarizes the findings of the research, and suggests next steps for this research.\item
\end{description}
