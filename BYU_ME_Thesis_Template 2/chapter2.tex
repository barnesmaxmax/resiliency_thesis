\chapter{Literature Review}
\label{chp:chapter2}
\graphicspath{{figures/}{figures/chapter2/}}

\section{Overview}

The resilience and connectivity of transport networks are a long-studied
topic within
transportation engineering in both theoretical and practical contexts.
Within this long history
however, there is variability in how scholars define resiliency. There are
three basic
definitions that researchers have used:

\begin{itemize}
	\item Resilience through Resistance: Resilient transportation networks
	have few and manageable vulnerabilities. This is typically addressed
	through robust facility-level engineering and risk management
	(e.g.\cite{bradley2007, peeta2010}).
	\item Resilience through Recovery: Resilient transportation networks are
	able to be repaired and returned to normal service without inordinate
	delay. This is accomplished through effective resource allocation and
	incident management during both disaster or degraded operation (e.g.,
	\cite{zhang2016}).
	\item Resilience through Operability in Crisis: Resilient transport
	networks are able to operate effectively with damaged or unusable links.
	It is this definition that is most relevant in the context of this study
	(\cite{berdica2002, ip2011}).
\end{itemize}

These definitions are not entirely mutually exclusive, and many
researchers apply more than one
definition in their work. For example, knowing where systemically critical
or vulnerable links
are will help in allocating maintenance resources. At the same time, the
approach to identifying
critical facilities implied by one of these definitions is not always
compatible with the other
definitions, and making distinctions between them is important
\cite{rogers2012}. A bridge
highly vulnerable to failure may be located on a little-traveled and
systemically unimportant
side street. The motivation of this research is to identify systemically
critical facilities, and
therefore we primarily consider literature using the third definition.

Professionals have adopted use of the Four R’s as a means to predict some
form of resilience on a
highway network. The Four R’s include: rapidity, redundancy, robustness,
and resourcefulness.
Here, rapidity is inversely related to the closure time and is used to
measure how quickly a road
can recover from a setback. Redundancy can be measured by the additional
time or distance a user
has to travel when a route is broken. Greater amounts of time or distance
lower the overall
redundancy. Robustness is the inverse of risk and represents the overall
strength of the system
as a whole. Resourcefulness, the last of the Four R’s, is the ability to
find quick solutions in
a network. An attempt will be made to identify the first time that these
terms surface in the
reviewed literature.

We begin this review first by examining the study conducted by AEM on
behalf of UDOT to identify
vulnerable sections on the I-15 corridor. We then consider observations
learned from systemic
changes to networks and populations under real-life crisis events. We then
consider previous
attempts in the academic literature to evaluate real and fabricated
transportation networks.

\section{Identifying Critical Links on I-15}

AEM worked with UDOT to develop an I-15 Corridor Risk and Resilience
(R\&R) Pilot \cite{aem2017}.
This
project had a seven-step plan to understand the impact of physical threats
to the Utah
transportation network, specifically looking at two sections along I-15.
These steps include
asset characterization, threat characterization, consequence analysis,
vulnerability assessment,
threat assessment, risk/resilience assessment, and risk/resilience
management. From these steps
AEM was able to provide recommendations to UDOT that would improve
resiliency along the corridor
based on the criticality of each segment at risk.

For the purpose of this study it is important to understand which threats
AEM decided to analyze.
For AEM to identify which threats to consider they needed, to use data
that was available to
them. AEM also ruled out certain threats based on their relevance. In the
end AEM considered nine
physical threats which include: earthquake, flood (scour), flood
(overtopping/debris), fire
(wildland), railway-proximity, oil/gas/water pipeline-proximity, and water
canal/ditch-proximity.
Then the available threat layers were intersected with assets (roadway,
bridge, etc.), and if
threat layers weren’t available historical data was analyzed to determine
their annual
probability.

Once these threat layers were determined and the location of the threat-
asset pairs along I-15
were found, AEM could then continue their analysis. This consisted of
gathering characteristic
data for each asset (length, width, depth, condition, etc.), determining a
replacement cost for
each asset, establishing an estimated service life for each asset,
estimating (if not known) the
design standard for each asset, establishing which magnitudes of each
threat were to be analyzed,
and gathering information on the likelihood of occurrence of each
magnitude of each threat. These
steps are further described in their paper using multiple different threat
locations.

The AEM R\&R report provides a good template going forward for identifying
links at risk,
following the first definition of a resilient transportation network. The
report also attempts to
identify which links at risk are most critical, assessing a “criticality”
score to the network
based on the five data elements and categories given in Table 1. A key
observation of these
criticality scores is that they do not accommodate alternate routes that
the traffic could use
should the link become unavailable. Identifying the systemic resiliency of
highway facilities
– as implied by the third definition of resiliency – requires considering
these alternate routes.

Insert table 1 here, AEM Criticality Score

\section{Lessons Learned from Crisis Events}

Two major crisis events in the last fifteen years have given researchers
an important opportunity
to study what happens to transportation behavior when critical links are
suddenly disabled for an
extended period of time. These events are the collapse of the I-35W bridge
collapse in
Minneapolis, Minnesota and the I-85 / Piedmont Road fire and bridge
collapse in Atlanta, Georgia.

\subsection{I-35W Bridge Collapse}

On August 1, 2007, the I-35 bridge over the Mississippi River in downtown
Minneapolis collapsed
during rush hour. The bridge, which was undergoing maintenance, had been
rated as structurally
deficient and fracture critical, meaning that failure of one member would
cause structure
failure. The collapse occurred during rush hour traffic, and the bridge
was additionally loaded
with approximately 300 tons of maintenance equipment \cite{schaper2017}.
There were 13
fatalities,
approximately 140 injuries, and abrupt disruption to roughly 140,000
average daily trips (ADT)
over the bridge \cite{zhu2010}. The complicated nature of the demolition
and repair meant
this systemically critical link would be missing for approximately 14
months. The approximate
location of the bridge, one of two major routes over the Mississippi
River, can be seen in Figure
1.

Insert Figure 1 here

\cite{zhu2010} conducts a travel survey to provide a more in-depth
analysis of important data
and traffic changes surrounding the I-35W bridge collapse in 2007. The
article uses a methodology
that attempts to identify mode-choice and other behavioral changes of
survey respondents. The
authors analyze data looking for variations in ADT, as well as Origin-
Destination matrices.
Importantly, they analyze some pre-disaster data in their work. The
authors provide evidence
which indicates that drivers are reluctant to make mode choice changes,
rarely doing so. This is
likely due to reasons such as finances, time, or perceived difficulty of
navigating a new mode of
transport. At the same time, some drivers change destinations when faced
with increased travel
times.

\cite{levinson2010} explore traffic behavior and changes in the wake of
major network
disruptions such as those that occurred in Minnesota. The authors identify
unique behavior post
disaster using GPS tracking data, survey data from the post disaster
phase, and other aggregate
data from surrounding freeways and traffic devices. The gathered data was
analyzed to track
changes in ADT over bridges and alternative routes in the area after the
disaster as well as
after mitigation was complete. Levinson and Zhu provide increased
understanding about how a
network's operability changes during a post-crisis environment.

\cite{xie2011} attempt to determine economic costs in the form of
increased travel time
of the 2007 I-35W bridge collapse using a scaled down travel demand model.
The authors used a
simplified version of the SONG 2.0 travel demand model that had been
developed for the Twin
Cities area to determine vehicle hours traveled (VHT) and vehicle
kilometers traveled (VKT). They
also calculate the accessibility for each zone from jobs to workers, and
from workers to jobs of
the network using employment, residency, and transportation cost data.
Using this simplified
travel model, the authors estimate that the bridge collapse cost the Twin
Cities approximately
\$75,000 per day in increased travel times.

\subsection{I-85/Piedmont Road Bridge Fire}

In Atlanta, Georgia, a section of an I-85 bridge collapsed due to a
massive fire under the bridge
on March 30, 2017. The fire, which was started by a homeless man, grew
quickly because of
improperly stored construction materials under the bridge. The approximate
location of the bridge
collapse caused by the fire can be seen in Figure 2; the damaged link is
at a critical point
downstream of a merge point between two expressway facilities (GA-400 and
I-85) bringing commuter
traffic in from the suburbs of northern Fulton and Gwinnett Counties.

INSERT iMAGE 2 HERE

As a result of the fire, the highway, which had an average daily traffic
count (ADT) of 243,000,
was closed in both directions for a period of about two months. This
closure led to a 30\%
increase in traffic volumes across the entire downtown network, with
increased congestion on side
streets \cite{hamedi2018}. Additionally, the Metropolitan Atlanta Rapid
Transit Authority
(MARTA), experienced a 20\% increase in ridership, likely because many
commuters made mode choice
and route changes. To mitigate this, headways between busses and trains
were decreased to allow
greater passenger volume. MARTA was able to add 142,000 rail miles, 1,100
train hours, 8,202 bus
miles, 512 bus hours, and 2,463 parking spaces in park and ride lots to
help further mitigate the
situation \cite{marta2017, marta2018}. It is likely that MARTA’s efforts
to mitigate passenger
volumes greatly influenced the onset of negative effects of the bridge
fire.

The section of I-85 that was closed impacted a large, upper income
demographic in the greater
Atlanta area who commuted across the bridge. This area was drastically
impacted by the disaster.
As a result, the Georgia Department of Transportation (GDOT) along with
the Governor created a
\$3.1 million incentive program to help motivate project completion ahead
of schedule. The bridge
was originally set to be closed for a period of 10 weeks, however, it re-
opened after just 6
weeks, with construction being completed a month ahead of schedule. The
accelerated finishing
date was estimated to have saved approximately \$27 million in user and
travel time costs
\cite{GDOT2017}. GDOT’s efforts to open the bridge quickly after its
collapse aided in abating negative user costs due to significant travel
time delays that surfaced
due to changes in route choice and assignment.

\subsection{Attempts to Evaluate Systemic Resiliency}

A number of researchers have conducted studies where they construct real
or fabricated
transportation networks, eliminate or degrade links in the network, and
evaluate the changes the
loss of these links introduced into some measure of network performance.
\cite{berdica2002}
atempts
to identify, define and conceptualize vulnerability by envisioning
analyses conducted with
several vulnerability performance measures including travel time, delay,
congestion,
serviceability and accessibility. Here, Berdica defines accessibility as
the ability for users to
travel between origins and destinations for any number of reasons. She
then uses the performance
measures to define vulnerability as the level of reduced accessibility due
to unfavorable
operating conditions on the network. In particular, Berdica identifies a
need for further
research toward developing a framework capable of investigating
reliability of transportation
networks.

In this section we will examine several attempts by numerous researchers
to do precisely this,
using different measures of network performance. A consolidation of this
discussion is summarized
in TABLE NAME HERE, namely the methods that different researchers have
used in examining network
performance under duress. The measures can be consolidated in to three
basic families:

\begin{itemize}
	\item \textbf{Network connectivity}: How does damage to a network
	diminish the connectivity
between network nodes?
	\item \textbf{Travel Time analysis}: How much do shortest path travel
	times between origins
and destinations increase on a damaged network?
	\item \textbf{Accessibility analysis}: How easily can the population
	using the damaged
network complete their daily activities?
\end{itemize}

The following sections discuss relevant studies in each group; Table
\ref{tab:authortable} consolidates these studies by year and labels them
with
an applicable group.

\begin{table}

\caption{\label{tab:authortable}Attempts to Evaluate Systemic Resiliency}
\centering
\begin{tabular}[t]{rll}
\toprule
Year & Author & Performance Metric\\
\midrule
2004 & Geurs and van Wee & Accessibility (isochrone, gravity, logsum)\\
2007 & Abdel-Rahim et al. & Network Connectivity\\
2008 & Taylor, M & Accessibility (logsum)\\
2010 & Peeta et al. & Travel time and cost\\
2010 & Geurs et al. & Accessibility (logsum)\\
\addlinespace
2010 & Levinson and Zhu & Travel time and cost\\
2010 & Zhu et al. & Travel time and cost\\
2011 & Agarwal et al. & Network connectivity\\
2011 & Ip and Wang & Network connectivity\\
2011 & Serulle et al. & Travel time and cost\\
\addlinespace
2011 & Ibrahim, S & Travel time and cost\\
2011 & Xie and Levinson & Accessibility (isochrone)\\
2013 & Omer et al. & Travel time and cost\\
2014 & Osei-Asamoah and Lownes & Network connectivity\\
2015 & Zhang et al. & Network connectivity\\
\addlinespace
2015 & Guze & Network connectivity\\
2015 & Jaller et al. & Travel time and cost\\
2015 & Xu et al. & Network connectivity\\
2016 & Winkler, C. & Accessibility (gravity)\\
2017 & Ganin et al. & Accessibility (gravity)\\
\addlinespace
2019 & Vodak et al. & Network connectivity\\
2019 & Hackl and Adey & Network connectivity\\
\bottomrule
\end{tabular}
\end{table}

\subsection{Network Connectivity}

Graph theory is the mathematical study of networks of nodes connected by
edges (links). Within
this discipline are the related concepts of network vulnerability and
connectivity that have been
accessed by researchers. In these studies, researchers tend to define
critical links as those
that connect to many other nodes (directly or indirectly), or as links
whose loss isolates a
number of nodes from the rest of the network.

\cite{guze2014} conducts a review of the known uses of graph theory before
reviewing several
other multi-criteria optimization methods. Guze’s methodology as it
relates to resiliency
involves an
analysis of the knapsack problem, however Guze tends to focus on flow
theory in transportation
systems as the best graph related analysis option. Graph theory is also
useful for identifying
shortest path, network connectivity, and other methods of network
optimization. Graph theory
supports the idea of resilience through recovery as well as operability
because of how it
represents networks with links and nodes, and its ability to identify next
shortest paths in the
case of disaster. Guze’s greatest contribution to transportation research
is a simplified method
for determining shortest path route options.

\cite{abdel2007} developed a multi-layered graph to examine the resiliency
of the traffic
signal control system in Boise, Idaho. The researchers determined which
traffic signals would be isolated by a failure to a particular power
substation,
and consequentially the percent of travel paths that would experience
diminished
levels of service. The research highlights the degree to which interrelated
infrastructure systems --- power, telecommunications, and transportation --
-
depend on each other, though the researchers did not attempt to look at the
connective resiliency of the transportation network directly.

\cite{agarwal2011} present a method to represent a transportation network
as a
hierarchical graph that can be analyzed more directly for vulnerabilities.
The
authors acknowledge, however, that a maximal failure consideration where a
node
is entirely isolated from the network is unlikely in a real-world network
with
multiple paths of connectivity.

Similarly, \cite{ip2011} address this shortcoming through the concept of
*friability*, or
the reduction of capacity caused by removing a link or node, in order to
determine criticality of individual links. The methodology relies on the
ability
to determine the weighted sum of the resilience of all nodes based on the
weighted average of connectedness with other city nodes in the network. The
authors determine that the recovery of transportability between two cities
largely depends on redundant links between nodes. The authors also comment
that
most traffic managers are more concerned with the friability of single
links
rather than the friability of multiple links or an entire system.

\cite{vodak2019} develop an approach to identify
critical links in a network by searching for the shortest independent
loops in the network. The algorithm progressively damages one or more links
between iterations to determine if nodes become isolated, or cut off from
the
network. If a node becomes easily isolated or has a higher likelihood of
becoming isolated, then there is a higher degree of vulnerability present
in the
network. This method can both identify critical links in individual
networks, as
well as provide a means to quantitatively compare networks.

\cite{osei2014} adopt a network analysis methodology that is able to
analyze
resilience using the mode of a transportation network. In this
article, the authors evaluate resilience by comparing the biological
network of a common mold
with a rail network. The author’s methodology uses global efficiency and
the size of the giant
component, or connected
components containing any number of a graph's vertices to determine
network efficiency using the
inverse of the shortest path while the giant component used to measure the
ratio of link
connections after disaster to those existing before disaster.

\cite{zhang2015.2} investigates the role of network topology, meaning the
layout of the network.
The authors provide several examples of network topology types including
hub and spoke, grid, and
ring networks. After computing resilience indexes, or general resilience
levels of each type of
network topology, Zhang determines that metrics such as throughput,
connectivity and average
reciprocal distance increase with greater linage, however they decrease as
networks become
larger. This is likely because larger networks have fewer node
connections, and
therefore are less redundant.

All of these graph theoretical approaches tend to break down to some
degree on
large, real-world networks where the number of nodes and links numbers in
the tens
of thousands, and the degree of connectivity between any arbitrary node
pair is
high.

\subsection{Changes in Travel Time}

Highway system network failures --- in most imaginable cases --- degrade
the
shortest or least cost path but typically do not eliminate it entirely.
The degree
to which travel time increases when a particular link is damaged could
provide an
estimate of the criticality of that link.

\cite{peeta2010} construct a model to efficiently allocate
highway maintenance resources. Each link in the sample network was
assigned a specific failure probability based on resource allocation;
the model evaluated the increase in travel time resulting from a broken
link. A Monte Carlo simulation revealed which allocation plan resulted in
the least
network degradation, and thus which links were most critical to the
network's
operations.

\cite{serulle2011} refines the previous work of other researchers to
define resiliency under
pre-event conditions. The authors accomplish this by clarifying variables
related to resiliency
of transportation networks including average delay and transport cost,
adjusting interactions,
and increasing metric transparency. The authors employ a methodology
capable of quantifying
resiliency using a fuzzy interference approach – an approach meant to use
imprecise or vague data
– that relates physical and performance characteristics. The used approach
is able to determine a
resiliency index that supports comparative and sensitivity analyses.
Accessibility data including
available road capacity, road density, alternate route proximity, average
delay, transport cost,
and average speed reduction are analyzed for importance to the integrity
of the network.

\cite{ibrahim2011} provide an alternative heuristic approach for
determining
vulnerability of infrastructure by estimating the cost of single link
failure
based on the increase in shortest path travel time due to increased
congestion
levels. The authors propose a hybrid heuristic approach that calculates the
traditional user-equilibrium assignment for finding the first set of
costs, and
then fixes those costs for all following iterations to determine the
effects of
failure on overall travel time of the system.  @omer2013 apply a similar
methodology
to a real-life intercity highway network. @jaller2015 extended this
methodology
with a static user equilibrium traffic loading step to provide an estimate
of
how the next-shortest path changes when congested.

\cite{omer2013} proposes a methodology for assessing the resiliency of
physical infrastructure
during disruptions. To do this, the authors use a network model to build
an origin-destination
matrix that allows initial network loading and analysis. Omer’s model uses
several metrics, but
the main metric used to determine resiliency is the difference in travel
time between a disturbed
and undisturbed network. Omer’s framework is applied to an actual network
between New York City
and Boston for analysis. Changes in demand, travel time, mode choice and
route choice are tracked
for analysis. Omer’s framework supports operability of transportation
networks due to the way it
analyzes networks experiencing suboptimal circumstances. The authors work
identifies key
parameters that should be measured to assess resiliency during disruptive
events.

\cite{jaller2015} seeks to identify critical infrastructure based on
increased travel time, or
reduced capacity due to disaster. The proposed methodology utilizes user-
equilibrium to determine
proper initial network loading. Then, the shortest path between one origin
and one destination
can be identified. To implement damage to the network, a link is cut, and
then the next shortest
path is found. This process is followed for all links in the system in
order to determine a sense
of the criticality of each link to network resiliency. The analysis is
carried out for each O-D
pair, and the nodes with greatest change in travel time are determined to
be the most critical.
Jaller’s methodology allows traffic managers to identify critical paths
for mitigation purposes
before the occurrence of disaster through careful analysis.

A primary limitation with increased travel time methodologies is that they
ignore the other possible ways a population might adapt its travel to a
damaged
network. Some people may choose other modes or destinations, and it is
possible
that some previously occurring trips might be canceled entirely.

\subsection{Changes in Accessibility}
In a travel modeling context, \textbf{accessibility} refers to the ease
with which
individuals can reach the destinations that matter to them; this is an
abstract
idea but one that has been quantified in numerous ways. \cite{dong2006}
provide a
helpful framework for understanding various quantitative definitions of
accessibility that we will simplify here. The most elementary definition of
accessibility is whether a destination is within an \textbf{isochrone}, or
certain
distance. This measure is often represented as a count, e.g., the number
of jobs
reachable from a particular location within thirty minutes travel time by a
particular mode. Mathematically,

\begin{equation}
A_i = \sum_{j} X_j I_{ij}; I_{ij} = \begin{cases}  1 \text{ if } d_{ij}
\leq D\\
0 \text{ if } d_{ij} > D \end{cases}
	\label{eqn:isochrone}
\end{equation}

where the accessibility \(A\) at point \(i\) is the sum of the all the
destinations \(X\) at other points \(j\). \(I_{ij}\) is an indicator
function equal to
zero if the distance between the points $d_{ij}$ is less than some asserted
threshold (e.g., thirty minutes of travel time). By relaxing the
assumption of a
binary isochrone and instead using the distance directly, we can derive the
so-called gravity model,

\begin{equation}
A_i = \sum_{j} X_j f(d_{ij})
  \label{eqn::gravity}
\end{equation}

where the function $f(d_{ij})$ is often a negative exponential with a
calibrated
impedance coefficient. An extension of the gravity model is to use the
logsum
term of a multinomial logit destination choice model,

\begin{equation}
A_i = ln\sum_{j} \beta_d(d_{ij}) + X_j\beta
  \label{eqn:logsum}
\end{equation}

where the parameters $\beta$ are estimated from choice surveys or
calibrated to
observed data. The logsum term has numerous benefits outlined by
\cite{handy1997}
and \cite{geurs2004}; namely, the measure is based in actual choice
theory, and
can include multiple destination types and travel times by multiple
different modes.

\cite{geurs2004} provide a review of accessibility measures such as those
above up to
2004. Of the papers they reviewed, [Vickerman, 1974, Ben-Akiva and Lerman,
1979, Geurs and
Ritsema van Eck, 2001] used isochrone type methods, [Stewart, 1947,
Hansen, 1959, Ingram, 1971,
and Vickerman, 1971, and Anas, 1983] used gravity style models, and
[Neuburger, 1971, Leonardi, 1987,
Williams and Senior, 1978, Koenig, 1980, Anas, 1983, Ben-Akiva and Lerman,
1985, Sweet,
1997, Niemier, 1997, Handy and Niemier, 1997, Levine, 1998, and Miller,
1999] used or
suggested logsums. They highlight the importance of using person-based
measures such as these in
evaluating network vulnerability and resiliency.

\cite{taylor2008} applied logsum-derived accessibility analysis to
evaluate the consequences of a tunnel failure in Adelaide, Australia. An
accessibility framework capable of evaluating the change in accessibility
for a multimodal urban network was desinged. The designed framework is
capable of determining the ability of an indivudal to access an activity
rather than travel between an OD pair (i.e. an activity-based model).

Taylor's framework captures five types of choice: activity, time period,
trip-base, location, and mode choice, with key features being activity
choice and trip-base (the origin point of a trip). Each of these choice
models use typical multinomial logit models (MNL), with the exception of
the mode choice model, which uses a netsted MNL model. The main choice
considered in the framework is activity choice followed by trip-base.

Taylor's proposed framework has been applied to an exisitng activity based
choice model for the Adelaide region, however, the framework operates
independently from the model in much the same way that the resiliency
model is independent from the USTM model. The resiliency model uses
similar inputs to the USTM model, but is a separate model.

Using the developed framework, Taylor calculates an "inclusive value" (IV)
and a "consumer surplus" (CS) valuse, which are similar to the utility
values calculated in the resiliency model. Both the IV and CS and the
utliity values are vital in determining the benefit or disbenefit
assicated with the change experienced by users in the individual model
scenarios.

In Taylor's model, the IV and CS values are estiamted using a logsum.
These values allow taylor to show that more disruption occurs near the
failed link than occurs farther away. Additionally, Taylor is able to show
that a greater cost (nearly 40 times greater) is experienced by those who
live in a TAZ near the link than by those who live in a TAZ located
farther away from the failed link. Taylor's framework primarily
investiates accessibility on a network for a large city,
but could easily be applied to a larger network, such as in the resileincy
model.

A key difference between the Adelaide and the resiliency model is that the
resileincy model calculates utility values to estimate the overall logsum
(benefit or disbenenfit caused by link failure) for each TAZ in the model
in much the same way Taylor uses the IV and CS values. However, the
resiliency model seeks to find the overall benefit or disbenefit
calculated as the cost associated with link closure accross the state,
rather than the local costs experienced at the TAZ level. The resiliency
model could easily be adapted to accomplish localized costs if needed.

In the Adelaide model, Taylor breaks one link and
then cacluates the difference in values using:

	\begin{equation}
		E(CS) = (1/\alpha) \log (\sum_{j = 1}^{J} \exp (I_j)) + \beta
			\label{eqn:taylor}
	\end{equation}

The logsum equation employed by Taylor is the exact same logsum used in
the resiliency model. Perhaps the most important distinction between the
two models is that the Adelaide model is an activity-based model while the
resiliency model is a trip-based model.

Taylors research highlights the need for a comprehensive model capable
of succinctly measuring the disbenefit caused by a degraded network.
Taylor continues by stating that traffic network simulation models
could be considered for future research. Some of the key needs for
future reseach specificaclly highlighted include:

	\begin{itemize}
		\item efficient algorithm development
		\item improved vulnerability metrics
		\item use of network vulnerability indicators in studies of critical
		infrastructure and the implications of network degradation
		\item improved techniques for identifying newtork weaknesses
	\end{itemize}

The resileincy model is a functional next step to Taylors research that
adresses some of the gaps mentioned above. Specifically, the resliency
model is desinged to work alongside a trip-based model (most statewide
travel modesl in the United States are trip-based \cite{tfr2021}), the
resiliency model provides simpler vulnerability metrics, studies and seeks
to identify techniques for identifying network weaknesses and important
links, and seeks to understand statewide effect of link failure.

In addition to the work discussed in this report, several other papers
expand on Taylor's appraoch in different yet important ways.

\cite{Jenelius2010} attempts to examine the importance of the link that
becomes critical only after partial network degradation, or redundancy
importance. This measure is primarily a flow-based measure.
\cite{Nassir2016} applies a nested logit model to examine a transit
network in Austrailia. The main contribution in \cite{Nassir2016} is an
improved methodology for calculating accessibiilty  measures related to
transit. One important observation however, is that users do not always
choose the fastest route, nor do they always choose the route with the
highest utility. \cite{He2012} takes another look at the after effects of
the I-35W bridge collapse previously discussed. A key contribution of
\cite{He2012} is that people often intially base route choice on what they
assume will be best based on past experience. So, over time, users will
adjust to an altered network.

\cite{Masiero2012} uses logit-based calculations to determine the Value of
Time (VOT) associated with the closure of a road in terms of cost, time,
and punctuality for freight transport. In order to properly determine the
VOT of route closure, \cite{Masiero2012} uses a method provided in
\cite{koppelman2006} which uses model derived coefficients and values to
determine the cost of an alternative. \cite{Masiero2012} implemented their
model on a network consisting of a single travel corridor that has
exprienced long (1 week to 2 months or more) closures in the past.

Taylor cites \cite{Berdica2007}, who attempt to examine the effects road
degradation has on Stockholm's transportation network if one or more
chokepoints were to become damaged or all-together innundated. The authors
sought to determine how interuptions affect the system, and how overall
system perfomrance was affected. Users in this method were only given the
choice of an alternate route, and the authors acknowledge that this is not
entirely reasonablein a real world situation. This method purely attempts
to quantify delay experienced by users compared to the original
equilibirum state, but does attempt to determine a moneatry value
associated with closure or degradation.

\cite{winkler2016} proposes a travel demand model that is valid for all
networks,
especially those with more than one constraint, such as mode choice.
Winkler’s methodology
utilizes an EVA model, or a
model that uses production, distribution, and mode choice as inputs. The
methodology shows that
an EVA model can be used to help determine outputs for multi-constraint
Multinomial Logit Models
(MNL) to help determine consumer surplus. Logsums like Winkler’s also
allow for consumer surplus,
measured as utility, to be estimated across the transportation network
being modeled.

\cite{ganin2017} attempts to investigate resiliency through a disruption
of 5\% of the roadways
in 40 urban networks within the United States. The employed methodology
determines that Salt Lake
City has the most resilient transportation network while Washington D.C.
has the least resilient.
This determination is based on a comparison of the network after links are
damaged versus before.
The authors work three factors into each model which account for
differences in car
truck ratios, average speed, and average vehicle length. Using a gravity
model, the authors were
able to estimate the average
annual delay per commuter. They used this to determine network efficiency.
Ganin noted that
traffic delay times increases significantly as road segments are broken.

\section{Summary}

The lessons learned from the events in Minneapolis and Atlanta demonstrate
that when
transportation networks are damaged or degraded by link failure, multiple
changes result. Traffic
diverts to other facilities and other modes, and some people make
fundamental changes to their
daily activity patterns, choosing new destinations or eliminating trips
entirely. Numerous other
researchers have identified methodologies to capture the effects, or at
least the costs, of these
potential changes in modeled crisis events. We are able to learn from
these methodologies to
create a methodology on a state-wide level. oogly moogly
