\chapter{Literature Review}
\label{chp:chapter2}
\graphicspath{{figures/}{figures/chapter2/}}

\section{Overview}

The resilience and connectivity of transport networks are a long-studied topic within
transportation engineering in both theoretical and practical contexts. Within this long history
however, there is variability in how scholars define resiliency. There are three basic
definitions that researchers have used:

\begin{itemize}
	\item Resilience through Resistance: Resilient transportation networks have few and manageable vulnerabilities. This is typically addressed through robust facility-level engineering and risk management (e.g., Bradley, 2007).
	\item Resilience through Recovery: Resilient transportation networks are able to be repaired and returned to normal service without inordinate delay. This is accomplished through effective resource allocation and incident management (e.g., Zhang and Wang, 2016).
	\item Resilience through Operability in Crisis: Resilient transport networks are able to operate effectively with damaged or unusable links. It is this definition that is most relevant in the context of this study. 
\end{itemize}

These definitions are not entirely mutually exclusive, and many researchers apply more than one
definition in their work. For example, knowing where systemically critical or vulnerable links
are will help in allocating maintenance resources. At the same time, the approach to identifying
critical facilities implied by one of these definitions is not always compatible with the other
definitions, and making distinctions between them is important (Rogers et al, 2012). A bridge
highly vulnerable to failure may be located on a little-traveled and systemically unimportant
side street. The motivation of this research is to identify systemically critical facilities, and
therefore we primarily consider literature using the third definition.

Professionals have adopted use of the Four R’s as a means to predict some form of resilience on a
highway network. The Four R’s include: rapidity, redundancy, robustness, and resourcefulness.
Here, rapidity is inversely related to the closure time and is used to measure how quickly a road
can recover from a setback. Redundancy can be measured by the additional time or distance a user
has to travel when a route is broken. Greater amounts of time or distance lower the overall
redundancy. Robustness is the inverse of risk and represents the overall strength of the system
as a whole. Resourcefulness, the last of the Four R’s, is the ability to find quick solutions in
a network. An attempt will be made to identify the first time that these terms surface in the
reviewed literature.

We begin this review first by examining the study conducted by AEM on behalf of UDOT to identify
vulnerable sections on the I-15 corridor. We then consider observations learned from systemic 
changes to networks and populations under real-life crisis events. We then consider previous
attempts in the academic literature to evaluate real and fabricated transportation networks. 

\section{Identifying Critical Links on I-15}
