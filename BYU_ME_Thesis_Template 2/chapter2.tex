\chapter{Literature Review}
\label{chp:chapter2}
\graphicspath{{figures/}{figures/chapter2/}}

\section{Overview}
\label{sec:ch2overview}

The resilience and connectivity of transport networks are a long-studied
topic within
transportation engineering in both theoretical and practical contexts.
Within this long history
however, there is variability in how scholars define resiliency. These definitions
could be classified into three categories:
\begin{itemize}
	\item \textbf {Resilience through Resistance}: Resilient transportation networks
	have few and manageable vulnerabilities. This is typically addressed
	through robust facility-level engineering and risk management
	\citep{bradley2007, peeta2010}.
	\item \textbf{Resilience through Recovery}: Resilient transportation networks are
	able to be repaired and returned to normal service without inordinate
	delay. This is accomplished through effective resource allocation and
	incident management during both disaster or degraded operation
	\citep{zhang2016}.
	\item \textbf{Resilience through Operability in Crisis}: Resilient transport
	networks are able to operate effectively with damaged or unusable links
	\citep{berdica2002, ip2011}.
\end{itemize}
It is the final definition of resilience that is most relevant in the context of this study.

These definitions are not entirely mutually exclusive, and many
researchers apply more than one
definition in their work. For example, knowing where systemically critical
or vulnerable links
are will help in allocating maintenance resources. At the same time, the
approach to identifying
critical facilities implied by one of these definitions is not always
compatible with the other
definitions, and making distinctions between them is important
\citep{rogers2012}. For example, a bridge
highly vulnerable to failure may be located on a little-traveled and
systemically unimportant
side street. Using the third definition a primary consideration, systemically critical
facilities can be identified.

This review begins by first examining a study conducted by AEM Corporation on
behalf of UDOT to identify
vulnerable sections on the I-15 corridor. Next, this review considers observations
learned from systemic
changes to networks and populations under real-life crisis events. Then, this review
considers previous
attempts in academic literature to evaluate the resiliency of real and fabricated
transportation networks.

\section{Identifying Critical Links on I-15}

AEM Corporation worked with UDOT to develop an I-15 Corridor Risk and Resilience
Pilot report \citep{aem2017}. This
project had a seven-step plan to understand the impact of physical threats
to the Utah
transportation network, specifically looking at two sections along I-15.
These steps included:

\begin{itemize}
	\item \textbf{Asset characterization}: A method to divide physical road assets into
	groupings with similar characteristics (e.g., roads, bridges, culverts, etc.)
	\item \textbf{Threat characterization}: A method to determine threat types each asset
	is exposed to or could be affected by (e.g., rock fall, fire, flood, etc.)
	\item \textbf{Consequence analysis}: An analysis determining the consequences of link
	loss, primarily estimating the cost of replacement should a link become
	damaged or broken
	\item \textbf{Vulnerability assessment}: An assessment of the amount of vulnerability
	each link is exposed to when single or multiple threat types are present
	\item \textbf{Threat assessment}: A method to determine the realized threat level
	present at each link examined
	\item \textbf{Risk/Resilience assessment}: A measure of the risk level and an attempt
	at a measure of the importance of each link to the roadway as a whole
	\item \textbf{Risk/Resilience management}: A summary of steps that should be taken to
	mitigate immediate risk, and reduce future risk while increasing the
	resilience level of individual road assets
\end{itemize}

From these different characterizations and assessments,
\citeauthor{aem2017} was able to provide a number of recommendations to UDOT that had the
potential to improve
resilience for the identified threat-asset pairs along the evaluated corridors (mainly sections of I-15)
based on the assigned criticality rating determined for each segment at risk.

It is easy to understand just how many natural or man-made threats exist to
current infrastructure. Natural disasters such as earthquakes, wildfire,
landslides and flash-floods cause billions of dollars of damage to infrastructure each year.
Other threats, such as terrorism, affect important infrastructure as well.
\citeauthor{aem2017} identified a number of threats, derived
from different types of data available for use. \citeauthor{aem2017} was
also able to rule out certain types of threats based on the relevance of these
threats in Utah. Ultimately, nine physical threat types were considered. These threats
include: earthquake, flood (scour), flood
(overtopping/debris), fire
(wild-land), railway-proximity, oil/gas/water pipeline-proximity, and water
canal/ditch-proximity. Data comprising historical disaster occurrences or
geographic location about these threat types exist and was assembled into
threat layers which were intersected with physical assets (e.g., roadway,
bridge, etc.).

Once these threat layers were determined and the location of the threat-
asset pairs along I-15
were found, \citeauthor{aem2017} was able to begin their analysis of how at-risk a link or road
segment might be to the nine identified threat types. This process consisted of
gathering characteristic
data for each asset (e.g., length, width, depth, condition, etc.), determining a
replacement cost for
each asset, establishing an estimated service life for each asset,
estimating (if not known) the
design standard for each asset, establishing which magnitudes of each
threat were to be analyzed,
and gathering information on the likelihood of occurrence of each
magnitude of each threat. These
steps are further described in the Risk and Resilience report published by \citet{aem2017}.

The \citeauthor{aem2017} report also provides a good template moving into the future for identifying
links at risk,
following the first definition of a resilient transportation network used in Section \ref{sec:ch2overview}. The
report also attempts to
identify which links are most critical, assessing a “criticality”
score to the network
based on the five data elements and categories given in Table \ref{tab:aemscore}.
Table \ref{tab:aemscore} provides insight into some of the complications involved in
attempting to identify critical roads. \citeauthor{aem2017} uses several classifications to group
roads together, such as the American Association of State Highway and Transportation
Officials (AASHTO) classification, and Average Annual Daily Traffic (AADT). For example, a road may have a low AADT, but
the majority of that AADT, which would show that there is a very low to low impact,
however, if the majority of traffic on that road were truck traffic, then that road almost
immediately has a moderate to high impact. Other nuances such as the one proposed in this
example exist. Another interesting situation to consider, is the case where a minor arterial becomes inundated with traffic or other hazard, however, a redundant arterial just a few blocks or miles away is able to handle much of the diverted traffic. Situations such as this one likely occur often, due to the way highway networks are traditionally built. One other observation made from Table
\ref{tab:aemscore}, is that \citeauthor{aem2017} does not take alternate routes into account.
Additionally, \citeauthor{aem2017} does not include a way for their risk analysis methodology
to anticipate what a user would actually do if faced with a real disaster
scenario. The work of \citeauthor{aem2017} does not answer simple questions such as how many valid
alternative routes exist? Or what is the new travel time or distance?
Identifying the systemic resiliency of highway facilities---as implied by the third definition of resiliency, resilience through
operability in crisis in Section \ref{sec:ch2overview}---requires considering these alternate routes \citep{aem2017}.

\begin{table}

\caption{AEM Criticality Score \citet{aem2017}}
\label{tab:aemscore}
\resizebox{\linewidth}{!}{
\centering
\begin{tabular}[t]{cccccc}
\toprule
Criteria & Very Low Impact & Low Impact & Moderate Impact & High Impact & Very High Impact\\
\midrule
AADT & $\leq 1,145$ & 1,146-3,275 & 3,276-8,285 & 8,286-17,455 & $>$17,455\\
\addlinespace
Truck AADT & 0 & 1-494 & 495-1,881 & 1,882-4,794 & $>$4,794\\
\addlinespace
\makecell{AASHTO\\Classification} & \makecell{Minor \\Collectors} & \makecell{Minor \\Collectors} & \makecell{Minor \\Arterials} & \makecell{Principal \\Arterials} & \makecell{Interstate \\Expressway}\\
\addlinespace
\makecell{Tourism\\Traffic (\$M)} & $<19.89$ & 19.90-39.66 & 39.67-101.13 & 101.14-505.32 & $>$505.32\\
\addlinespace
\makecell{Maintenance\\Distance (Miles)} & $<70$ & 71-84 & 85-102 & 103-124 & $>124$\\
\bottomrule
\end{tabular}
}
\end{table}

\section{Lessons Learned from Crisis Events}

Two major crisis events in the last 15 years have given researchers
an important opportunity
to observe and study what happens to transportation networks and user behavior when critical links are
suddenly disabled for an
extended period of time. These events are the I-35W bridge
collapse in
Minneapolis, Minnesota, and the I-85/Piedmont Road fire and bridge
collapse in Atlanta, Georgia. Both of these events were studied post-disaster,
when the highway network was already operating in crisis, allowing researchers
to examine how networks and road users operate or behave on a damaged network.

\subsection{I-35W Bridge Collapse}

On August 1, 2007, the I-35 bridge over the Mississippi River in downtown
Minneapolis collapsed
during rush hour. The bridge, which was undergoing maintenance, had been
rated as structurally
deficient and fracture critical, meaning that failure of one member would
cause catastrophic structure failure \citep{schaper2017}.
The collapse occurred during rush hour traffic, and the bridge
was additionally loaded
with approximately 300 tons of maintenance equipment \citep{schaper2017}.
There were 13
fatalities,
approximately 140 injuries, and abrupt disruption to roughly 140,000
average daily traffic (ADT)
over the bridge \citep{zhu2010}. The complicated nature of the demolition
and repair meant
this systemically critical link would be missing for approximately 14
months. The approximate
location of the bridge, one of two major routes over the Mississippi
River, can be seen in Figure \ref{fig:i35}.

\begin{figure}

{\centering \includegraphics[width=0.75\linewidth]{figures/chapter2/I-35W.png}

}

\caption{Approximate location of the I-35W bridge collapse.}\label{fig:i35}
\end{figure}

\citet{zhu2010} conducted a travel survey that provided a more in-depth
analysis of important data
and traffic changes surrounding the I-35W bridge collapse in 2007. The
study attempted to identify mode-choice and other behavioral changes of
survey respondents after the collapse. The
authors analyzed data looking for variations in ADT, as well as changes in overall origin-destination (OD) matrices.
Importantly, they analyzed loop detector data, bus ridership statistics, and survey response data in their work. The
authors conducted a regression analysis of the collected data,
which indicated that drivers are reluctant to make mode choice changes,
rarely doing so in the real world. This is
likely due to finances, time, or perceived difficulty of
navigating a new mode of
transport. At the same time, some drivers easily change destinations or routes when faced
with increased travel
times.

\citet{levinson2010} explored traffic behavior and changes in the wake of
major network
disruptions such as those that occurred in Minnesota. The authors identified
unique behavior, post
disaster, using GPS tracking data, survey data from the post disaster
phase, and other aggregate
data from surrounding freeways and traffic devices. These data were
analyzed to track
changes in ADT over bridges and alternate routes in the area after the
disaster as well as
after mitigation was complete. The authors provided increased
understanding about how a
network's operability changes in a post-crisis environment.

\citet{xie2011} attempted to determine economic costs in the form of
increased travel time
of the 2007 I-35W bridge collapse using a scaled-down travel demand model.
The authors used a
simplified version of the SONG 2.0 travel demand model that had been
developed for the Twin
Cities area to determine vehicle hours traveled (VHT) and vehicle
kilometers traveled (VKT). They
also calculated the accessibility for each zone, from jobs to workers, and
from workers to jobs, of
the network using employment, residency, and transportation cost data.
Using this simplified
gravity-based travel demand model, the authors estimated that the bridge collapse cost the Twin
Cities approximately
\$75,000 per day in increased travel times. Interestingly, they are able to
show that accessibility between workers and jobs was heavily affected by the
loss of the bridge. The ease with which road users can access locations around the region
experienced a dramatic decrease on the crippled network when compared to the whole, unbroken network.

\subsection{I-85/Piedmont Road Bridge Fire}

In Atlanta, Georgia, a section of bridge along I-85 near Piedmont Road collapsed due to a
massive fire under the bridge
on March 30, 2017. The fire grew
quickly because of
improperly stored construction materials under the bridge. The approximate
location of the bridge
collapse caused by the fire can be seen in Figure \ref{fig:i85}; the damaged link was
at a critical point
downstream of a merge point between two expressway facilities (GA-400 and
I-85) bringing commuter
traffic in from the suburbs of northern Fulton and Gwinnett Counties.

\begin{figure}
\begin{center}

{\centering \includegraphics[width=0.65\linewidth]{figures/chapter2/I-85.png}}

\caption{Approximate location of the I-85/Piedmont Road bridge fire.}\label{fig:i85}

\end{center}
\end{figure}

The section of I-85 that was closed impacted a large, upper income
demographic in the greater
Atlanta area who commuted across the bridge.
As a result, the Georgia Department of Transportation (GDOT) along with
the Governor created a
\$3.1 million incentive program to help motivate project completion ahead
of schedule. The bridge
was originally scheduled to be closed for a period of 10 weeks, however, it re-opened after just 6
weeks, with construction being completed a month ahead of schedule. The
accelerated finishing
date was estimated to have saved approximately \$27 million in user and
travel time costs
\citep{GDOT2017}. GDOT’s efforts to reconstruct the bridge and quickly reopen the highway after the
bridge collapse and immediate highway closure aided in abating negative user costs (or dis-benefit) due to significant travel
time delays that surfaced (in a post-disaster environment)
due to changes in route choice and assignment.

As a result of the fire, the highway, which had an ADT of 243,000,
was closed in both directions for a period of about 6 weeks. This
closure led to a sudden 30\%
increase in traffic volumes across the entire downtown network, with
notable increased congestion on side
streets \citep{hamedi2018}. Additionally, the Metropolitan Atlanta Rapid
Transit Authority
(MARTA) experienced a 20\% increase in ridership, likely because many
commuters made mode choice
and route changes. To mitigate this, headways between buses and trains
were decreased to allow
greater passenger volume. MARTA was also able to extend service capacity by about 20\%, adding 142,000 rail miles, 1,100
train hours, 8,202 bus
miles, 512 bus hours, and 2,463 parking spaces in park and ride lots to
help further mitigate the sudden increase in ridership \citep{marta2017, marta2018}. It is likely that MARTA’s efforts
to mitigate the rapid increase in passenger
volumes greatly reduced any negative effects of the bridge
fire on transit services, and helped alleviate other congestion generated by the disaster.

\subsection{Observations: Change in Route, Mode, and Destination}
Several lessons and key takeaways exist from both the I-35W bridge collapse and the
I-85/Piedmont Road bridge fire and collapse. First, researchers observed that users
quickly make route changes when faced with abruptly altered networks, but are reluctant
to make mode choice changes \citep{levinson2010}. Second, researchers observed
route and mode changes in the days and weeks following the Atlanta I-85 bridge fire.
The effects of these changes were offset by efforts made by MARTA to extend transit
services \citep{hamedi2018, marta2017, marta2018}. These real world observations
are invaluable because both route and mode choice changes do occur, and should therefore
be considered in future transportation planning efforts.

\section{Attempts to Evaluate Systemic Resiliency}

Real world events do occur; however, and it is important for researchers to
base efforts on both theoretical scenarios, and on actual events.
Thus, a number of researchers have conducted studies where real
or fabricated transportation networks are constructed, damaged or degraded, and
then changes in performance measures are evaluated. All of this is done to measure
network performance without
an actual disaster occurring beforehand.

\citet{berdica2002}
attempted
to identify, define and conceptualize vulnerability by envisioning
analyses conducted with
several vulnerability performance measures, including travel time, delay,
congestion,
serviceability and accessibility. Here, Berdica defined accessibility as
the ability for users to
travel between OD pairs for any number of reasons. She
then used the performance
measures to define vulnerability as the level of reduced accessibility due
to unfavorable
operating conditions on the network. In particular, Berdica identified a
need for further
research toward developing a framework capable of investigating or measuring
the overall systemic resiliency of transportation
networks.

In the following section, the work of several researchers who have attempted to develop
a framework capable of investigating or measuring the overall systemic resiliency
of transportation networks will be examined. Many of the researchers use different
methods to measure network performance while a network is operating under some kind of duress,
and a consolidation of this
discussion is summarized
in Table \ref{tab:authortable}. The measures can be consolidated into three
basic areas of study:

\begin{itemize}
	\item \textbf{Network connectivity}: How does damage to a network
	diminish the connectivity
between network nodes?
	\item \textbf{Travel Time analysis}: How much do shortest path travel
	times between origins
and destinations increase on a damaged network?
	\item \textbf{Accessibility analysis}: How easily can the population
	using the damaged
network complete their daily activities? This in turn can be evaluated in a number of ways as explained by \citet{dong2006}.
\end{itemize}

The following sections discuss relevant studies in each group; Table
\ref{tab:authortable} consolidates these studies by year and labels them
with
an applicable group.

\begin{table}[h]

\caption{\label{tab:authortable}Attempts to Evaluate Systemic Resiliency}
\centering
\begin{tabular}[t]{lll}
\toprule
Author & Year & Performance Metric\\
\midrule
Geurs and Van Wee & 2004 & Accessibility (isochrone, gravity, logsum)\\
Dong et al. & 2006 & Accessibility\\
Abdel-Rahim et al. & 2007 & Network Connectivity\\
Berdica and Mattsson & 2007 & Network Connectivity\\
Taylor & 2008 & Accessibility (logsum)\\
Peeta et al. & 2010 & Travel time and cost\\
Geurs et al. & 2010 & Accessibility (logsum)\\
Zhu et al. & 2010a & Travel time and cost\\
Zhu et al. & 2010b & Travel time and cost\\
Agarwal et al. & 2011 & Network Connectivity\\
Ip and Wang & 2011 & Network Connectivity\\
Serulle et al. & 2011 & Travel time and cost\\
Ibrahim et al. & 2011 & Travel time and cost\\
Xie and Levinson & 2011 & Accessibility (isochrone)\\
He and Liu & 2012 & Travel time and cost\\
Masiero and Maggi & 2012 & Accessibility \\
Omer et al. & 2013 & Travel time and cost\\
Osei-Asamoah and Lownes & 2014 & Network connectivity\\
Guze & 2014 & Network connectivity\\
Zhang et al. & 2015 & Network connectivity\\
Jaller et al. & 2015 & Travel time and cost\\
Xu et al. & 2015 & Network connectivity\\
Nassir et al. & 2016 & Accessibility \\
Winkler & 2016 & Accessibility (gravity)\\
Ganin et al. & 2017 & Accessibility (gravity)\\
Vodák et al. & 2019 & Network connectivity\\
\bottomrule
\end{tabular}
\end{table}

\subsection{Network Connectivity}

Graph theory is the mathematical study of networks of nodes connected by
edges (links), and is useful for identifying shortest paths on a network.
Concepts closely related to graph theory, such as network
vulnerability and network connectivity, have been studied by researchers who
have considered resilience. In these studies, researchers tend to define
critical links as those
that are well connected to many other nodes (directly or indirectly), or as links
whose loss easily isolates a
number of nodes from the rest of the network \citep{west2001}.

\citet{abdel2007} developed a multi-layered graph approach to examine the resiliency
of the traffic
signal control system in Boise, Idaho. The researchers determined which
traffic signals would be isolated by a failure to a particular power
substation,
and consequentially the percent of travel paths that would experience
diminished
levels of service. The research highlighted the degree to which interrelated
infrastructure systems---power, telecommunications, and transportation---depend on each other, though the researchers did not attempt to look at the
connective resiliency of the transportation network directly.

\citet{agarwal2011} presented a method to represent a transportation network
as a
hierarchical, or cluster graph, that can be analyzed more directly for
vulnerabilities. Clusters are formed as groups of links and nodes become
isolated from each other. These clusters of links and nodes are then grouped
together more tightly by including nearby clusters, which creates a
``zoomed out'' or less complex model where small clusters begin to act as nodes connected by links.
In the study, links in the system are damaged, and the resulting
connectedness of the network is evaluated. One scenario of importance
noted by the authors, however, is that a maximal failure consideration where a
node
is entirely isolated from the network is unlikely in a real-world network
with
multiple paths of connectivity. The authors discussed the importance of
having damaged networks with high levels of functionality.

\citet{vodak2019}
on the other hand, developed an approach to identify
critical links in a network by searching for the shortest independent
loops in the network. An independent loop is essentially a way to travel
between an origin and a destination over any number of alternative routes.
The algorithm progressively damaged one or more links
between iterations to determine if nodes become isolated, or cut off from
the
network. If a node becomes easily isolated or has a higher likelihood of
becoming isolated, then there is a higher degree of vulnerability present
in the
network. This method can both identify critical links in individual
networks and provide a means to quantitatively compare networks.

\citet{ip2011} addressed the concept of {friability}, or
the reduction of capacity caused by removing a link or node, in order to
determine criticality of individual links. The methodology relied on the
ability
to determine the weighted sum of the resilience of all nodes based on the
weighted average of connectedness with other city nodes in the network. The
authors determined that the recovery of transportability between two cities
largely depends on redundant links between nodes. The authors also commented
that
most traffic managers are more concerned with the friability of single
links
rather than the friability of multiple links or an entire system. This suggests
that planners and managers may not consider the importance of
understanding the impacts of widespread, all-inclusive disaster scenarios.

\citet{guze2014} conducted a review of the known uses of graph theory before
reviewing several other multi-criteria optimization methods.
Guze’s methodology involved an
analysis of the knapsack problem which focused on flow theory in transportation
systems and identifying a method to find the best graph solution. Guze’s
greatest contribution to transportation research
at the time was a simplified method
for determining shortest path route options on simple networks.

\citet{osei2014} adopted a network analysis methodology that is able to
analyze
resilience of transportation networks. In this
article, the authors evaluated resilience by comparing the biological
network of a common mold
with a rail network. The network for both the mold and the railway are
complexly connected. The ``giant component'', a construct of a graph representing its connectivity, is given by:
\begin{equation}
	\Phi = \frac{E'}{ E}
\end{equation}
where $\Phi$ represents the giant component, $E$ represents the
level of connectivity before the network is damaged, and $E'$ represents the
level of network connectivity after the network is damaged. After the giant component
is found, network efficiency is determined using the
shortest path available. By combining both the ratio of link connections and
network efficiency, the authors draw comparisons between two
complex networks. Ultimately, the authors concluded that a denser, more highly
interconnected network would perform better when a link is cut due to a larger giant component value.

\citet{zhang2015} investigated the role of network topology, or the physical
layout of the network in a geographic location.
The authors provided several examples of network topology types including
hub and spoke, grid, and
ring networks. After computing resilience indexes, or general resilience
levels of each type of
network topology, the authors determined that metrics such as throughput,
connectivity and average
reciprocal distance increased with greater lineage, however these same measures decreased as
networks became
more widespread. This is likely because larger networks typically have fewer, less dense node
connections, and
therefore are less redundant.

Each of the graph theoretical approaches discussed in this section tend to break
down in efficiency or connectivity as networks become larger. Real-world
networks are typically extremely large, with nodes and links numbering in the
hundreds, if not thousands. Thus, the connectivity of a node may be high, but may
not be an accurate representation of a node's importance. Lack of node importance
can be due to several factors, including network efficiency \citep{osei2014}, network topology \citep{zhang2015},
friability \citep{ip2011}, or other factors.

\subsection{Changes in Travel Time}

Highway system network failures---in most imaginable cases---degrade
the
shortest or least cost path, but typically do not eliminate all paths.
The degree
to which travel time increases when a particular link is damaged could
provide an estimate of the criticality of that link or node. If a link or node
becomes completely isolated, the travel time to that node or link would
increase indefinitely.

\citet{Berdica2007} attempted to examine what the effects of road
degradation on Stockholm's transportation network would be if one or more
chokepoints were to become damaged or all-together inundated. The authors
sought to determine how interruptions affect the system, and how overall
system performance was affected by the damaged or lost facility. Users in this method were only given the
choice of an alternate route, and the authors acknowledge that this is not
entirely reasonable in a real world situation because users would likely change
departure or arrival time, mode, or even destination. This method purely attempted
to quantify delay experienced by users compared to the original
equilibrium state, but does make an attempt to determine a monetary value
associated with closure or degradation.

\citet{peeta2010} constructed a model to efficiently allocate
highway maintenance and emergency response resources at locations throughout a
network. Each link in the sample network was
assigned a specific failure probability based on resource allocation;
the model evaluated the increase in travel time resulting from a broken
link. The authors applied a Monte Carlo simulation of multiple scenarios,
which revealed resource allocation plans with the least network degradation,
and thus which links were most critical to the network's operations.

\citet{ibrahim2011} provided an approach for determining
vulnerability of infrastructure by estimating the cost of single link
failure based on the increase in shortest path travel time due to
increased congestion
levels. The authors proposed a hybrid heuristic approach that calculates the
traditional user-equilibrium assignment for finding the first set of
costs, and
then fixes those costs for all following iterations to determine the
effects of
failure on overall travel time of the system.

\citet{omer2013} proposed a methodology for assessing the resiliency of
physical infrastructure
during disruptions. To do this, the authors used a network model to build
an origin-destination
matrix which allows initial network loading and analysis. Omer’s model used
several metrics, but
the main metric used to determine resiliency is the difference in travel
time between a disturbed
and undisturbed network. Omer’s framework is applied to an actual network
between New York City
and Boston for analysis. Changes in demand, travel time, mode choice and
route choice are tracked
for analysis. Omer’s framework supports operability of transportation
networks (as seen in Section \ref{sec:ch2overview}) due to the way it
analyzes networks experiencing suboptimal circumstances. Omers's work
identified key
parameters that should be measured to assess resiliency during disruptive
events including mode and route choice.

\citet{jaller2015} sought to identify critical infrastructure based on
increased travel time or
reduced capacity due to disaster. The proposed methodology utilized user-equilibrium to determine
proper initial network loading. Then the shortest path between one origin
and one destination
could be identified. To implement damage to the network, a link was cut, and
then the next shortest
path was found. This process is followed for all links in the system in
order to determine a sense
of the criticality of each link to network resiliency. The analysis is
carried out for each OD
pair, and the nodes which experience the greatest change in travel time are determined to
be the most critical.
Jaller’s methodology allowed traffic managers to identify critical paths
for mitigation purposes
before the occurrence of disaster through careful analysis.

\citet{ganin2017} attempted to investigate resiliency through a disruption
of 5\% of the roadways
in 40 urban networks within the United States. The employed methodology
determined that Salt Lake
City had the most resilient transportation network while Washington D.C.
had the least resilient.
This determination is based on a comparison of simple gravity models of the identified networks after links are
damaged versus before.
The authors worked three factors into each model, which account for
differences in car-
truck ratios, average speed, and average vehicle length. Using a gravity
model to determine commuting patterns, the authors were
able to estimate the average
annual delay per commuter. They used this to determine network efficiency.
\citeauthor{ganin2017} noted that
traffic delay times (or the travel time caused by a closure) increased
significantly as road segments were broken.

A primary limitation with increased travel time methodologies is that they
ignore other possible ways a population might adapt its travel to a
damaged
network. Some people may choose other modes or destinations, and it is
possible
that some previously occurring trips might be canceled entirely. It may also be
prudent to consider how access changes, and evaluate changes in user choice
based purely upon accessibility.

\subsection{Changes in Accessibility}
\label{sec:cacc}
In a travel modeling context, {accessibility} refers to the ease
with which
individuals can reach the destinations that matter to them; this is an
abstract
idea but one that has been quantified in numerous ways. \citet{dong2006}
provide a
helpful framework for understanding various quantitative definitions of
accessibility that we will simplify here. The most elementary definition of
accessibility is whether a destination is within an isochrone, or
destinations accessible within the same distance or time interval from an origin.
This measure is often represented as a count (e.g., the number
of jobs
reachable from a particular location within 30 minutes travel time by a
particular mode). Mathematically, accessibility is defined as:
\begin{equation}
A_i = \sum_{j} X_j I_{ij}; I_{ij} = \begin{cases}  1 \text{ if } d_{ij}
\leq D\\
0 \text{ if } d_{ij} > D \end{cases}
	\label{eqn:isochrone}
\end{equation}

\noindent where the accessibility \(A\) at point \(i\) is the sum of the all the
destinations \(X\) at other points \(j\). \(I_{ij}\) is an indicator
function equal to
zero if the distance between the points $d_{ij}$ is less than some asserted
threshold (e.g., 30 minutes of travel time). By relaxing the
assumption of a
binary isochrone and instead using the distance directly, we can derive the
so-called gravity accessibility model as:
\begin{equation}
A_i = \sum_{j} X_j f(d_{ij})
  \label{eqn::gravity}
\end{equation}

\noindent where the function $f(d_{ij})$ is an impedance function, often a
negative exponential with a calibrated
impedance coefficient. This gravity accessibility term is included in the
gravity trip distribution model (i.e., the gravity model). An extension of the gravity model is to use the
logsum
term of a multinomial logit destination choice model:
\begin{equation}
A_i = ln\sum_{j} \beta_d(d_{ij}) + X_j\beta
  \label{eqn:logsum1}
\end{equation}

\noindent where the parameters $\beta$ are estimated from choice surveys or
calibrated to
observed data. The logsum term has numerous benefits outlined by
\citet{handy1997}
and \citet{geurs2004}; namely, the measure is based in actual choice
theory, and
can include multiple destination types and travel times by multiple
different modes.

\citet{geurs2004} provide a review of accessibility measures such as those
above, up to
2004 in a literature review done at the time. Of the papers they reviewed, \citet{vickerman1974}, \citet{ben1979}, \citet{geurs2001}, used isochrone type methods. \citet{stewart1947}, \citet{hansen1959}, \citet{ingram1971}, \citet{vickerman1974}, \citet{anas1983}, used gravity
style models, and \citet{neuburger1971}, \citet{leonardi1978}, \citet{williams1978}, \citet{koenig1980indicators}, \citet{anas1983}, \citet{ben1985discrete}, \citet{sweet1997aggregate}, \citet{niemeier1997accessibility}, \citet{handy1997}, \citet{levine1998rethinking}, \citet{miller1999measuring} used or
suggested logsums. These researchers highlight the importance of using person-based
measures such as logsums in
evaluating network vulnerability and resiliency.

\citet{taylor2008} applied logsum-derived accessibility analysis to
evaluate the consequences of a tunnel failure in Adelaide, Australia. An
accessibility framework capable of evaluating the change in accessibility
for a multimodal urban network was designed. The designed framework is
capable of determining the ability of an individual to access an activity.
Taylor's framework captured five types of choice, namely: activity, time period,
trip-base, location, and mode choice, with key features being activity
choice and trip-base (e.g., the origin point of a trip). Each of these choice
models use typical multinomial logit models (MNL), with the exception of
the mode choice model, which uses a nested MNL model. The main choice
considered in the framework is activity choice followed by trip choice.
Taylor's proposed framework has been applied to an existing activity based
choice model for the Adelaide region, however, the framework operates
independently from the parent model.

Using the developed framework, Taylor calculated an ``inclusive value'' (IV) or logsum,
and ``consumer surplus'' (CS) or utility value. Both the IV and CS values are vital to
determining the benefit or dis-benefit
associated with the change experienced by users in the individual model
scenarios considered. Taylor's accessibility framework, applied as a separate or connected module to
the Adelaide model, estimates the IV and CS values using a logsum.
These values allowed Taylor to show that more disruption occurs near the
failed link than occurs farther away. Additionally, Taylor is able to show
that a greater cost (nearly 40 times greater) is experienced by those who
live in a TAZ near the link than by those who live in a TAZ located
farther away from the failed link. Taylor's framework primarily
investigated accessibility on a network for a large city,
but could easily be applied on a larger scale.

In the Adelaide model, Taylor breaks one link and
then calculates the difference in IV and CS values using:
	\begin{equation}
		I_n=\log \sum_{r \in R} \exp({V_{rn}})
	\end{equation}
which represents the IV or logsum value, and estimates the CS value using:
	\begin{equation}
		E(CS) = \frac{1}{\alpha} \log (\sum_{j = 1}^{J} \exp (I_j)) + \beta
			\label{eqn:taylor}
	\end{equation}
\noindent where \(\alpha\) represents the negative of the coefficient of time or cost from the utility function,
and \(\beta\) is an unknown constant that represents the difference between the actual value of CS and the estimated value.
The \(I_j\) term represents the observable attributes of the possible utility.

Taylor's research highlighted the possibility for a comprehensive model capable
of succinctly measuring the dis-benefit caused by a degraded network.
Taylor continued by stating that traffic network simulation models
could be considered for future research.  Four key needs for
future research specifically highlighted at the conclusion of Taylor's article include:

	\begin{itemize}
		\item {Development of an efficient algorithm}
		\item {Development of improved vulnerability metrics}
		\item {Improved techniques for identifying network weaknesses}
		\item {Use of network vulnerability indicators in studies of critical
		infrastructure and the implications of network degradation}
	\end{itemize}

Several other authors employed various types of logit models in their research,
or attempted to develop a methodology specifically for use in analyzing disrupted networks.
\citet{geurs2010} compared the benefits attributed to various climate
change-mitigation land use and transport policies under two different
evaluation metrics. The authors directly compared a “rule of half”
calculation where the travel time reduction is distributed between existing
and new travelers on a route with a mode choice logsum derived from utility
theory. The authors argued that because the logsum is more comprehensive and
inclusive of the full changes to travel demand (capturing the total value of
the choice set), the additional benefits attributed to proposed projects are
both more realistic and more economically favorable to climate change
mitigation policies. They showed this by evaluating the accessibility and
travel time changes resulting from land use densification strategies in the
national travel demand model for the Netherlands.

\citet{serulle2011} clarified variables related to resiliency
of transportation networks including average delay and transport cost,
adjusting interactions,
and increasing metric transparency. The authors employed a methodology
capable of quantifying
resiliency using a fuzzy interference approach---an approach meant to analyze
imprecise or vague data---that relates both physical and performance characteristics. The employed approach
is able to determine a
resiliency index that supports comparative and sensitivity analyses.
Accessibility data, including
available road capacity, road density, alternate route proximity, average
delay, transport cost,
and average speed reduction, are analyzed for importance to the integrity
of the network.

\citet{Masiero2012} used logit-based calculations to determine the value of the
indirect costs associated with the closure of a road in terms of economic consequences, including
punctuality for freight transport. In order to properly determine the
cost of route closure, the authors also used a method discussed in
\citet{koppelman2006} which used model derived coefficients and values to
determine the cost of an alternative. The authors implemented their
model on a network consisting of a single travel corridor that has
experienced long (1 week to 2 months or more) closures in the past in Southern Europe.

\citet{xiangdong2015} developed network-based measures and computational
methods to evaluate transportation network redundancy. This methodology used
two dimensions in the analysis: travel alternative diversity and network spare
capacity, meaning the number of effective connections available for each OD
pair and the congestion effect or choice behavior of travelers. To create the
analysis, the authors first constructed a sub-network which only looks at
efficient routes. Next, the method counted the number of efficient routes
from the origin to all nodes, and then estimates the multi-modal network
spare capacity. Each dimension helps evaluate the capacity of the network
based on different scenarios. \citeauthor{xiangdong2015}'s method supports operability
(seen in Section \ref{sec:ch2overview}) by evaluating the capacity of
transportation networks using altered or damaged networks.

\citet{Nassir2016} applied a nested logit model to examine a transit
network in Australia. The main contribution in \citet{Nassir2016} is an
improved methodology for calculating accessibility measures related to
transit, accomplished by developing a method to account for diversity
information. The authors did note, however, that this measure is best applied
to models with complex transit systems that serve large portions of the
community. One important observation, however, is that users did not always
choose the fastest route, nor did they always choose the route with the
highest utility. \citet{He2012} took another look at the after effects of
the I-35W bridge collapse previously discussed. A key contribution of
\citet{He2012} is that people often initially base route choice on what they
assume will be best based on past experience. So, over time, users
adjusted their route choice to an altered network. The true implication here
is that users did not automatically choose the best new route given an
altered network, rather, it takes users time to learn how the new network
functions.

\citet{winkler2016} proposed a travel demand model that is valid for all
networks,
especially those with more than one constraint, where a constraint is a limiting factor
present in the model, such as the maximum number of trips produced at an origin, or
the maximum number of receivable trips at a destination. A logsum approach fails when two
or more sets of constraints exist. The proposed method uses CS, a measure of benefit,
to derive a method capable of analyzing a doubly constrained network. Traditionally,
constraints are used for trip distribution, and the best known model is the doubly constrained
gravity model. Winkler’s methodology utilized a
model that uses production, distribution, and mode choice as inputs. The
methodology shows that models can be used to help determine outputs for
multi-constraint MNLs. Winkler uses the change in CS as the logsum difference, which
would allow utility to be estimated across the transportation network
being modeled.

\section{Summary}

The lessons learned from the events in Minneapolis and Atlanta demonstrate
that when
transportation networks are damaged or degraded by link failure, multiple
changes result. Traffic
diverts to other facilities and other modes, and some people make
fundamental changes to their
daily activity patterns, choosing new destinations or eliminating trips
entirely. Numerous other
researchers have identified methodologies to capture the effects, or at
least have made quality attempts to capture the costs of these
potential changes to accessibility in modeled crisis events.
From this extensive review of existing literature, we are able to see that no
one has attempted evaluate the relative systemic
criticality of highway links on a statewide network using a logit-based model
sensitive to changes in route path, destination choice, and mode choice. Doing so would
provide a method for determining network vulnerabilities using estimates of dis-benefit
that consider the entire available choice set, and estimates which account for user choice.
