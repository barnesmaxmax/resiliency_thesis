% This is just to make sure that if it goes onto multiple pages that
% it will only be on odd pages.
\afterpage{\cleardoublepage}
% Some people have had problems with needing a little more space or a little less space right before the body of the abstract. If you have that problem, you can uncomment the next line, and either add or take away space manually (negative spaces are OK).
%\vspace*{-0.05in}

The objective of this study is to evaluate the relative systemic
criticality of highway links on a statewide network using a logit-based model
sensitive to changes in route path, destination choice, and mode choice.
In recent history, transportation network vulnerabilities have been increasingly
scrutinized. Transportation disasters, such as the collapse of the I-35W
bridge in Minneapolis, Minnesota, and the I-85 / Piedmont Road fire
and subsequent bridge collapse in Atlanta, Georgia, have brought the need to easily identify
network vulnerabilities to the forefront of the Utah Department of Transportation's
(UDOT) planning efforts. UDOT manages and maintains a complex statewide network of highways,
made up of facilities which include bridges, mountain passes, and canyon roads.
To help with transportation planning efforts, UDOT has developed the Utah
Statewide Transportation Model (USTM), which is a trip-based model. However,
UDOT does not currently possess a model capable of quickly identifying network
vulnerabilities, and the potential costs -- measured as dis-benefit -- associated with link loss.
Subsequently, this study seeks to evaluate the relative systemic
criticality of highway links on a statewide network using a logit-based model
sensitive to changes in route path, destination choice, and mode choice.
Home-based Work (HBW), Home-based Other (HBO), and Non-home Based (NHB)
trips were primarily considered in the development of what has been named the
Resiliency Model. Logit-based calculations are advantageous for two reasons: first,
logsums capture the total value of user choice, and second, logsums are able to easily
calculate total accessibility based on the value of the available choice set. Thus, logsum
calculations were used as the main input into a mode- and destination choice
model, which uses a logsum calculation to determine the dis-benefit experienced
by road users. The ability of logsum calculation to incorporate user choice
along with multiple other data types, makes these modeling calculations
advantageous compared to traditional modeling methods. In this study, transit trips were held
fixed, as they are outside the scope of the project. The results of this study
are made up of the dis-benefit experienced by road users measured using the
change in destination choice logusm, converted to cost measured in dollars per day.
