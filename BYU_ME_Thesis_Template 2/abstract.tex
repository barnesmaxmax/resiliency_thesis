% This is just to make sure that if it goes onto multiple pages that
% it will only be on odd pages.
\afterpage{\cleardoublepage}
% Some people have had problems with needing a little more space or a little less space right before the body of the abstract. If you have that problem, you can uncomment the next line, and either add or take away space manually (negative spaces are OK).
%\vspace*{-0.05in}

In recent history, transportation network vulnerabilities have been increasingly
scrutinized. Recent transportation disasters, such as the collapse of the I-35W
bridge in Minneapolis, Minnesota, and the I-85 / Piedmont Road fire
and bridge collapse in Atlanta, Georgia, have brought the need to easily identify
network vulnerability to the forefront of the Utah Department of Transportation's
(UDOT) planning efforts. UDOT manages and maintains a complex statewide network of highways,
made up of facilities which include bridges, mountain passes, and canyon roads.
To help with transportation planning efforts, UDOT has developed the Utah
Statewide Transportation Model (USTM), which is a trip-based model. However,
UDOT does not currently possess a model capable of quickly identifying network
vulnerabilities, and the potential costs associated with link loss.
Subsequently, this study seeks to develop a trip-based
choice model using the USTM network, combined
with socioeconomic data from the Utah Household Travel Survey (UHTS) taken in
2017, to estimate the dis-benefit experienced by road users if a link becomes
damaged. Home-based Work (HBW), Home-based Other (HBO), and Non-home Based (NHB)
trips were primarily considered in the development of what has been named the
Resiliency Model. User mode
choice, with options for automobile, non-motorized, and transit modes, was first
found using a logsum calculation. Logit-based calculations are advantageous for two reasons: first,
they capture the total value of user choice, and second, logsums are able to easily
calculate total accessibility based on the value of choice. Thus, logsum
calculations were then used as the main
input into a destination choice model which also uses a logsum to
determine a user's final destination choice. Adapting a trip-based model,
such as USTM, to a logit-based destination choice model was the primary goal of
this thesis. The ability of logit calculation to incorporate user choice along
with multiple other data types, makes these modeling calculations advantageous
when compared with
traditional modeling methods. The other goal of this thesis is to determine the
dis-benefit experienced by road users as a result of link loss on the USTM
network. In this study, transit trips were held
fixed, as they are outside the scope of the project.
