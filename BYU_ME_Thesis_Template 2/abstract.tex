% This is just to make sure that if it goes onto multiple pages that
% it will only be on odd pages.
\afterpage{\cleardoublepage}
% Some people have had problems with needing a little more space or a little less space right before the body of the abstract. If you have that problem, you can uncomment the next line, and either add or take away space manually (negative spaces are OK).
%\vspace*{-0.05in}


The Utah Department of Transportation (UDOT) manages and maintains a complex
state-wide network of highways. Recent incidents such as the collapse of the I-35W bridge in Minneapolis, Minnesota, and the I-85/Piedmont Road fire and subsequent bridge collapse in Atlanta, Georgia, have brought identification of transportation network vulnerabilities to the forefront of UDOT’s planning efforts. Traditional estimates of transportation network impacts have focused on increases to user travel time or the volume of affected traffic, but studies of these disasters have revealed that when facing a degraded transportation network, people adjust their trip making in terms of destination, mode, and route choice. The objective of this thesis is to evaluate the relative systemic criticality of highway links on Utah’s highway network using a logit-based model sensitive to changes in destination choice, mode choice, and route path. The current Utah Statewide Travel Model (USTM) does not incorporate user mode or destination choice, making it unsuitable for this task in its present condition. Consequently, this thesis develops a logit- based model structure that evaluates the cost of impaired destination choices and mode choices for home-based and non-home-based personal trips resulting from a damaged highway network. The choice model logsums capture the total value of user choices and can be readily converted to monetary values, making them ideal for this purpose. The logit-based model is then applied to 40 highway links located at strategic locations on Utah’s network. When compared with a more traditional travel time increase estimation, the logsum and travel time models provide categorically different cost estimates, where the logsum results are typically lower than travel time estimates, with implications for policy making and UDOT’s planning strategy. The results further suggest that freight trips are likely more important considerations than passenger trips, and should be considered in future research.
