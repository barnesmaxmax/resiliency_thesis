% This is just to make sure that if it goes onto multiple pages that
% it will only be on odd pages.
\afterpage{\cleardoublepage}
% Some people have had problems with needing a little more space or a little less space right before the body of the abstract. If you have that problem, you can uncomment the next line, and either add or take away space manually (negative spaces are OK).
%\vspace*{-0.05in}

In recent history, transportation network vulnerabilities have been increasingly
scruitinized. Recent transportation disasters, such as the collapse of the I-35W
bridge collapse in Minneapolis, Minnesota, and the I-85 / Piedmont Road fire
and bridge collapse in Atlanta, Georgia, have brought the need to identify
network vulnerability to the forefront of the Utah Department of Transporation's
planning efforts. Subsequently, this study sought to develop a trip-based
choice model using the Utah State Trasnportation Model (USTM) network, combined
with socioeconomic data from the Utah Household Travel Survey (UHTS) taken in
2017. Home-based Work (HBW), Home-based Other (HBO), and Non-home Based (NHB)
trips were primarily considered in the development of what has been named the
Resiliency Model. These trip purposes constitute the majority of trips which
do not have fixed origins or destinations, such as freight trips. User mode
choice, with options for automobile, non-motorized, and transit modes, was first
found using a logsum calculation. These results were then used as the main
input into a destination choice model which also uses a logsum calculation to
determine a user's final destination choice. To determine the differences 
between the base model and alternative scenarios, a link was removed from the
USTM network for each alternative, and then the results were compared against
the base model. This compariosn returns the overall change in logsum, which is
a measure of the benefit or disbenefit experienced by a user in the model.
The final logsum value returned was then adapted into a cost value which
determines the overall cost experienced by users on the USTM network for each
scenario. The model provides a more sensitive analysis estimate of the costs
experienced by road users in the case of link loss than would a traditional
model or non-choice based estimate. In this study, transit trips were held
fixed, as they are outside the scope of the project.
