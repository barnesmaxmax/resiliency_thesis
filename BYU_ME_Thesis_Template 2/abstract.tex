% This is just to make sure that if it goes onto multiple pages that
% it will only be on odd pages.
\afterpage{\cleardoublepage}
% Some people have had problems with needing a little more space or a little less space right before the body of the abstract. If you have that problem, you can uncomment the next line, and either add or take away space manually (negative spaces are OK).
%\vspace*{-0.05in}


%PURPOSE
The objective of this thesis is to evaluate the relative systemic
criticality of highway links on a statewide network using a logit-based model
sensitive to changes in route path, destination choice, and mode choice.
%VULNERABILITIES EXIST & WAY TO IDENTIFY LINKS
In recent history, transportation network vulnerabilities have been increasingly
scrutinized. Transportation disasters, such as the collapse of the I-35W
bridge in Minneapolis, Minnesota, and the I-85/Piedmont Road fire
and subsequent bridge collapse in Atlanta, Georgia, have brought identification of
vulnerabilities to the forefront of transportation planning efforts.
The Utah Department of Transportation (UDOT) manages and maintains a complex statewide network of highways.
Currently, UDOT does not possess a method capable of quickly identifying network
vulnerabilities, or potential costs associated with extended road closure.
%USTM DOESN'T ACCOUNT FOR
To aid planning efforts, UDOT has developed the Utah
Statewide Transportation Model (USTM), which is a trip-based gravity model. USTM
estimates the number of trips on Utah's highway network, however, USTM does not
incorporate user mode or destination choice.
Consequently, USTMs results are inflexible, meaning that trips cannot change mode or destination.
Incorporating choice flexibility into USTM will allow human behavior to be represented.
%PRESENT A LOGIT MODEL - COSTS
Subsequently, this thesis seeks to evaluate the relative systemic
criticality of highway links on a statewide network using a logit-based model
sensitive to changes in route path, destination choice, and mode choice.
Logit-based calculations are advantageous in transportation modeling for two reasons: first,
logsums capture the total value of user choice, and second, logsums are able to calculate
total accessibility based on the value of the available choice set. The logsum model measures the overall
dis-benefit experienced on an altered network, and this change is then
converted into a measure of cost in dollars per day associated with
road closure.
%COMPARE TO A TRAVEL TIME MODEL
Additionally, a travel time model was built to compare the logsum results to
the results from traditional modeling methods.
The travel time model calculates the total change in travel time by trip purpose,
and converts into a measure of cost in dollars per day associated with road closure.
%RESULTS
The results of this thesis show that the logsum and travel time models
provide different cost estimates, where the logsum results are typically lower
than travel time estimates. Freight trips, which are included in
only the travel time model, return estimated costs which far exceed the estimated
costs for any other trip purpose or purposes combined.
%WHAT DOES THIS DO FOR US
The development of the presented logsum model is important for two reasons:
first, the model provides different estimates when compared with traditional methods, and
second, the model accounts for route change and user mode and destination choice on a statewide network.
Incorporating a logsum model into USTM allows UDOT to better identify
vulnerabilities and estimate costs associated with road closure while considering user choice.
